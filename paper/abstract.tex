\begin{abstract}

%% Problem / Motivation [1s]
Modern keyless door security systems remain largely inaccessible to the public, due to their high cost and low transparency.
%% Solution (key idea) [2s]
Our system, SpoonPi, is a cost-effective  solution that is designed to be intuitive to use, and easy to customize or extend. It supports five different authentication mechanisms involving three factors: RFID, fingerprint and PIN.
%% More detail on contribution [2s]
The single-door security provided by SpoonPi easily scales to multiple doors via the accompanying web app, ForkPi, which provides secure, centralized management of multiple SpoonPis over a local network. \\
By choosing Raspberry Pi as the platform, we are able to open source the code, and make the system easy to replicate.
%% Some evidence (evaluation results) [2s]
In this paper, we discuss the details of our design and implementation, and evaluate our system in terms of security, usability, and cost-effectiveness. We find that the system possesses a viable level of security, users find it easy to use, and it is reasonably priced for a system of its capability.

\end{abstract}

%There is a growing need for secure, cost-effective hardware authentication systems. To answer to that need, the study aims to build a Raspberry Pi-based authentication system that is inexpensive, open-source, and customizable while maintaining a comparable level of security.

%Five authentication mechanisms involving the RFID, Fingerprint, and PIN were implemented for the system and are gauged using the metrics of security, cost, and usability. The system was evaluated using several security criteria such as its lifetime against brute force attacks, and vulnerability to key forgery or theft. The cost of the system was then compared to the observed cost of similar systems sold commercially. In conclusion, the system possesses a viable level of security and cost-effectiveness.
